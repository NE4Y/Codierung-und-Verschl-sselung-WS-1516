\documentclass[a4paper, openany]{book}
\usepackage{titlesec}
\usepackage[utf8x]{inputenc}
\usepackage[document]{ragged2e}
\usepackage{enumitem}
\usepackage{filecontents}
\usepackage{tikz}
\usepackage{amsmath, amssymb, amstext, amsfonts, mathrsfs}
\usepackage{qtree}
\setcounter{MaxMatrixCols}{20}
\renewcommand{\chaptername}{}
\renewcommand{\contentsname}{Inhaltsverzeichnis}
\titleformat{\chapter}
  {\Large\bfseries} % format
  {}                % label
  {0pt}             % sep
  {\huge}           % before-code

\author{Steffen Lindner}
\title{\vspace{-2cm}Codierung und Verschlüsselung}

\begin{document}
\maketitle
\tableofcontents



\chapter{Kryptologie}

\foreach \c in {1,2,3,4,5,6} {\input{Kryptologie/krypto\c.tex}}


\chapter{Codierungstheorie}

\foreach \c in {1,2} {\input{Codierung/cod\c.tex}}

\section{Syndrom-Decodierung linearer Codes}

$\mathcal{C}$ lin. [n,k]-Code über K, $|k| = q$, Kontrollmatrix H, (n-k) $\times$ n -Matrix. Ist $y \in K^n$, so heißt $H \cdot y^t \in K^{n-k}$ \underline{Syndrom} von $y$

\begin{enumerate}[label=(\alph*)]
	\item $x \in \mathcal{C} \Leftrightarrow H \cdot x^t = 0 \Leftrightarrow x$ hat Syndrom 0

	\item $y_1, y_2$ liegen in der gleichen Nebenklassen bzgl. $\mathcal{C} \Leftrightarrow H \cdot y_1^t = H \cdot y_2^t$ (d.h. $y_1, y_2$ haben gleiches Syndrom)


	[$y_1 \in y_2 + \mathcal{C} \Leftrightarrow y_1 - y_2 \in \mathcal{C} \Leftrightarrow H \cdot y_1^t - H \cdot y_2^t = H(y_1-y_2)^t = 0 \Leftrightarrow H \cdot y_1^t = H \cdot y_2^t$]

	\item Jedes $z \in K^{n-k}$ tritt als Syndrom auf.

	($rg(H) = n-k =$ Rang der lin. Abb. $\begin{cases}y^t \rightarrow H \cdot y^t \\ K^n \rightarrow K^{n-k} \end{cases}$)
\end{enumerate}

Situtation: $x \in \mathcal{C}$ wird gesendet. $y = x + f$ wird empfangen, $f$"Fehlervektor".

Nach (b):

\begin{center}
	$y$ und $f$ haben das gleiche Syndrom
\end{center}

Hamming-Decodierung:

$y$ gegeben. Suche $x' \in \mathcal{C}$ mit min. Hamming-Abstand zu y. Das heißt: Suche e mit min. Gewicht, so dass $y-e \in \mathcal{C}$.

\par \medskip

e von min. Gewicht in der Nebenklasse $y + \mathcal{C}$, d.h. $e$ von minimalem Gewicht mit $H \cdot y^t = H \cdot e^t$.

\par \medskip

Wähle in jeder Nebenklasse ienem Vektor e von minimalem Gewicht: \underline{Nebenklassenführer} 

\par \medskip

Liste der Nebenklassenführer.

Ordne Syndrome.

Binär: lexikografisch.

$\begin{pmatrix}0 \\ ... \\ 0 \end{pmatrix} \leftrightarrow$ Nebenklasse ($\mathcal{C}$) notiere Nebenklasse f.

usw.

\par \medskip

$y$ empfangen. Berechne $H \cdot y^t$.

\par \medskip

Suche Nebenklassenführer zu diesem Syndrom. Decodiere $y \rightarrow y-f \in \mathcal{C}$

\par \medskip

Syndrom-Decod. $2^{n-k}$ Nebenklassenführer $\in \mathbb{Z}_2^n$.

Speicherbedarf: $n \cdot 2^{n-k}$ Bit

\par \medskip

Liste aller Codewörter: $n \cdot 2^k$ Bit Speicher

\par \medskip

n = 70, k = 50 $\rightarrow$ Syndrom-Decod.: $70 \cdot 2^{20}$ Bit $\approx$ 8,75 MB

Liste aller Codewörter: $70 \cdot 2^{50}$ Bit $\approx$ 9 PB

\chapter{Beispiele guter linearer Codes}

\section{Hamming-Codes}

$q$ Primzahlpotenz, $K$ Körper mit $|K| = q$.

Sei $l \in \mathbb{N}, l \ge 2$ mit \[ n = \frac{g^l -1}{q-1} \in \mathbb{N} \] $k = n-l$.

$(q = 2, n= 2^l -1$)

\par \medskip

Dann existiert perfekter [n,k]-Code über K, $d(\mathcal{C}) = 3$, \underline{Hamming-Code}.

\subsection{Konstruktion}

Es gilt:

\[|K^l \setminus \{\sigma\}| = q^l -1 \] 

Deswegen gibt es in $K^l$ genau $\frac{q^l -1}{q-1}$ 1-dim. Unterräume.

\par \medskip

Wähle aus jedem 1-dim. Unterraum von $K^l$ einen Vektor $\neq 0$ aus. Schreibe diese als Spaltenvektoren in eine Matrix H. H ist $l \times n$ - Matrix.

\par \medskip

\textbf{Klar:} $rg(H) = l$, denn $rg(H) \le l$, da H nur $l$ Zeilen enthält. $rg(H) \ge l$, da H $l$ linear unabhängige Spalten enthält.

\par \medskip

$\mathcal{C}$ = Code mit H als Kontrollmatrix.

$\mathcal{C} = \{y \in K^n : H \cdot y^t = 0 \}$ \underline{Hamming-Code}

\par \medskip

$dim(\mathcal{C}) = n-l = k$, $|\mathcal{C}| = q^k$.

\par \medskip

Je 2 Spalten von $H$ sind linear unabhängig (denn sonst kämen sie aus demselben 1-dim. Unterraum von $K^l$). Es gibt 3 linear abhängige Spalten in $H$:

\[ s_1 = \begin{pmatrix}a \\ 0 \\ 0 \\ 0 \end{pmatrix}, s_2 = \begin{pmatrix}0 \\ b \\ 0 \\ 0 \end{pmatrix}, s_3 = \begin{pmatrix}c \\ c \\ 0 \\ 0 \end{pmatrix}, a,b,c \neq 0 , \frac{c}{a} s_1 + \frac{c}{b} s_2 - s_3 = \sigma\]

10.13: $d(\mathcal{C}) = 3$. 1 - Fehler-korrigierender Code (t=1)

\par \medskip

Perfekter Code:

$K^n = \bigcup_{c \in \mathcal{C}} K_t(c)$ (disjunkt)

\par \medskip

Kugelpackungsschranke mit Gleichheit:

\[|\mathcal{C}| \cdot (1 + n \cdot (q-1)) \overset{!}{=} q^n \]

\[ q^{n-l} \cdot (1 + q^l -1) = g^n \]

Damit ist Hamming-Code perfekter Code.

\par \medskip

Bei festem $q$ und $l$ gibt es viele Möglichkeiten für H (und damit für $\mathcal{C}$). 

\begin{itemize}
	\item Auswahl der Vektoren $\neq 0$ aus 1-dim. Unterraum von $K^l$.
	\item Reihenfolge der Spalten in H.
\end{itemize}

Führt jeweils zu perfekten Codes mit gleichen Parametern ("äquivalent").

\section{Beispiel}

\begin{enumerate}[label=(\alph*)]
	\item $q = 2, K = \mathbb{Z}_2 = \{0,1 \}, l =3, n = 2^3 -1 = 7, k = n-l = 4$.

	[7,4] - Code über $\mathcal{Z}_2$.

	\[ H = \begin{pmatrix}1 & 0 & 0 & 1 & 0 & 1 & 1\\ 0 & 1 & 0 & 1 & 1 & 0 & 1 \\ 0 & 0 & 1 & 0 & 1 & 1 & 1 \end{pmatrix} \]

	[7,4] - Hamming-Code zu $H$ über $\mathcal{Z}_2$ = Bsp. aus 9.12.d)

	\item $q = 3, l =3, n = \frac{3^3 -1}{3-1} = 13$

	\[ H = \begin{pmatrix}1 & 0 & 0 & 1 & 2 & 0 & 0 & 1 & 2 & 1 & 1 & 1 & 2 \\ 0 & 1 & 0 & 1 & 1 & 1 & 2 & 0 & 0 & 1 & 1 & 2 & 1 \\ 0 & 0 & 1 & 0 & 0 & 1 & 1 & 1 & 1 & 1 & 2 & 1 & 1 \end{pmatrix} \]

	$dim(\mathcal{C}) = 10, |\mathcal{C}| = 3^{10} = 59.049$

	\par \medskip

	Anwendung: Toto, Tippe alle Codewörter aus $\mathcal{C} \rightarrow$ mind. einmal 12 oder 13 Richtige.


	\item $[7,4]$ - Hamming-Code über $\mathbb{Z}_2$ aus a).

	Ist $y = (1110110) \in \mathcal{C}$?

	\[ H \cdot y^t = \begin{pmatrix} 0 \\ 0 \\1 \end{pmatrix} \neq \begin{pmatrix}0 \\0 \\ 0 \end{pmatrix}, \text{ d.h. y $\not \in \mathcal{C}$} \]

	$\mathcal{C}$ perfekt: Es gibt genau ein Codewort $x \in \mathcal{C}$ mit $d(x,y) = 1$.

	Wie findet man $x$? Es geht schneller als Syndrom-Decodierung:

	\par \medskip

	$x$ unterscheidet sich von $y$ an einer Stelle: $y_i \neq x_i$.

	\par \medskip

	$y = x + (0,...,0,\underbrace{1}_{i},0,...,0)$.

	Bilde $H \cdot y^t = Hx^t + H \cdot \begin{pmatrix}0 \\ ... \\ 1 \\ 0 \\ ... \\ 0 \end{pmatrix}$ = i-te Spalte von H.

	\par \medskip

	Prüfe an welcher Stelle i Syndrom $H \cdot y^t$ als Spalte in Kontrollmatrix auftritt. Ändere $y$ an Stelle i.

	\par \medskip

	In unserem Beispiel $i = 3$. $y \rightarrow x = (1100110) \in \mathcal{C}$
\end{enumerate}

\section{Decodierung von Hamming-Codes}

Für $\mathbb{Z}_2$ wie in $11.2.c)$. Lässt sich auf bel. Körpern verallgemeinern.

\section{Definition}

K (endl.) Körper. $\left \langle , \right \rangle$, $K^n \times K^n \rightarrow K$

\par \medskip

$u = (x_1, ..., x_n), v = (y_1, ..., y_n) : \left \langle u,v \right \rangle = \sum_{i=1}^n x_i y_i$

\section{Bemerkung}

\begin{enumerate}[label=(\alph*)]
	\item $\left \langle u, v+w \right \rangle = \left \langle u,v  \right \rangle + \left \langle u,w \right \rangle$

	\item $\left \langle u, a \cdot v \right \rangle = a \cdot \left \langle u,v \right \rangle$

	Analog im 1. Argument.

	\item $\left \langle u, v \right \rangle = \left \langle v, u \right \rangle$

	\item $\left \langle 0, v\right \rangle = \left \langle v, 0 \right \rangle = 0$

	\item Ist $u \in K^n$ mit $\left \langle u, v \right \rangle = 0$ für alle $v \in K^n$, so ist $u = 0$.
\end{enumerate}

$(a)-(c): \left \langle , \right \rangle$ ist symmetrische Bilinearform.

\subsection{Beweis}

(e):

$0 = \left \langle u, e_i \right \rangle = x_i, i=1, ..., n$ mit $u = (x_1, ..., x_n)$

\par \medskip

\textbf{Beachte:} Aus $\left \langle v, v \right \rangle = 0$, folgt nicht $v = 0$.

\par \medskip

Zum Beispiel: $\mathbb{Z}_2: \left \langle v, v \right \rangle = 0 \Leftrightarrow \sum_{i=1}^n y_i^2 = 0$

\par \medskip

$v = (y_1, ..., y_n) : \sum_{i=1}^n y_i = 0 \Leftrightarrow$ Anzahl der Einsen in v ist gerade ($\mathbb{Z}_2$)





\end{document}


  