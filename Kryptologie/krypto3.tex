\chapter{Der Advanced Encryption Standard (AES)}

\textbf{1970'er:} Entwickeltdes DES (auf der Basis einer Feistelchiffre). Blocklänge n = 64, effektive Schlüssellänge 56 Bit. Anfällig gegen Brute-Force-Angriffe.

\par \medskip

\textbf{1997:} Ausschreibung durch NIST für neuen Verschlüsselungsstandard 

\par \medskip

\textbf{2000:} Rijndael-Verfahren (J. Daemen, V. Rijmen)

\par \medskip

\textbf{2002:} AES 

\section{Struktur des AES}

AES ist iterierte Blockchiffre.

Blöcklängen: 128, 192, 256 Bit

Schlüssellängen: 128, 192, 256 Bit 

Blocklängen unabhängig von Schlüssellängen

Anzahl der Runden (abhängig von gewählten Bitlängen), r= 10, 12, 14

\par \medskip

Zwischenergebnisse nach jeder Runde:

Zustände (States), $S_0, ..., S_9$.

Jede Runde besteht aus 4 Transformationen:

\par \medskip

$S_i \rightarrow$ Sub Bytes $\rightarrow $ Shift Rows $\rightarrow $ Mix Columns $\rightarrow \oplus K_{i+1} \rightarrow S_{i+1}$ (Bild)

\par \medskip

Mix Columns fällt in der letzten Runde weg.

\textbf{Vorbemerkung}

128 Bit-Blöcke: $4 \times 4$ - Matrizen, jeder Eintrag Byte.  $\begin{pmatrix}a_{00} & a_{01} & a_{02} & a_{03} \\ a_{10} & a_{11} & a_{12} & a_{13} \\ a_{20} & a_{21} & a_{22} & a_{23} \\ a_{30} & a_{31} & a_{32} & a_{33} \end{pmatrix} \equiv$  Block ($a_{00}a_{10}a_{20}a_{30}a_{01}...a_{33}$) (Spaltenweise auslesen).

\par \medskip

Bytes werden häufig als Elemente in einem Körper der Ordnung $2^8$ aufgefasst (Anzahl der Elemente).

\par \medskip

\textbf{Beispiel} für endlichen Körper: $\mathbb{Z}_p$ = $\{0,...,p-1\}$, p Primzahl

\begin{equation}
  a \oplus b = a+b \mod p
\end{equation}

\begin{equation}
  a \odot b = a \cdot b \mod p
\end{equation} 

$\mathbb{Z}_2 = \{0,1\}$, $\oplus$ XOR, $\odot$ $\cdot$, $\wedge$

Körper der Ordnung $2^8: \mathbb{F}_{2^8}$

Elemente sind alle Polynome vom Grad $<$ P über $\mathbb{Z}_2$:

\begin{equation}
  a_7x^7+...+a_1x+a_0 \leftrightarrow (a_7, a_6, ..., a_0), \ \ a_i \in \mathbb{Z}_2
\end{equation}

Addition in $\mathbb{F}_{2^8}$: Addition von Polynomen.

Multiplikation in $\mathbb{F}_{2^8}$: $f,g \in \mathbb{F}_{2^8}$

\begin{equation}
  f \odot g := f \cdot g \mod h
\end{equation}

wobei h das irreduzible Polynom $h = x^8 + x^4 + x^3 + x + 1$

\par \medskip

\textbf{Beispiel:} 

$(x^7 + x + 1) \odot (x^3 + x) = x^{10} + x^8 + x^4 + x^3 + x^2 + x \mod h = x⁶ + x⁵ +x³ + 1$

In $\mathbb{F}_{2^8}$ existiert zu jedem $0 \neq g \in \mathbb{F}_{2^8}$ ein $g^{-1} \in \mathbb{F}_{2^8}$ mit $g \circ g^{-1} = 1$

Berechnung mit dem erweiterten Euklidischen Algorithmus (wie in $\mathbb{Z}$).

\begin{equation}
  0 \neq f, g \in \mathbb{Z}_2[x]: u,v \in \mathbb{Z}_{2}[x] \textnormal{ mit } u \cdot f + v \cdot g = ggT(f,g)
\end{equation}

$0 \neq g \in \mathbb{F}_{2^8}, h$ irred. Polynom. ggT(g,h) = 1, da h irred.

\par \medskip

EEA: $u \cdot h + v \cdot g = 1, u,v \in \mathbb{Z}_{2}[x]$

$(v \mod h) \odot g = (v \cdot g) \mod h = 1$

\par \medskip

\textbf{(Verkürtzter) EEA in $\mathbb{Z}_2[x]$:}


Input: $f,g \in \mathbb{Z}_2[x], g \neq 0$, Grad(f) $\geq$ Grad(g)

Output: $ggT(f,g), v \in \mathbb{Z}_2[x]: v \cdot g \mod f = ggT(f,g)$:

\begin{enumerate}
  \item $s := f$, $t:=g$, $v_1 := 0$, $v_2 := 1$, $v:=1$
  \item Solange $s \mod t \neq 0$ wiederhole

  $q:= s \textnormal{ div } t, r := s \mod t$

  $v := v_1 - q \cdot v_2$, $v_1 := v_2$, $v_2 := v$

  $s := t, t := r$

  \item Output: $t = ggT(f,g), v$
\end{enumerate}

Zur Bestimmung von $g^{-1}$ für $0 \neq g \in \mathbb{F}_{2^8}$ wende EEA an mit $f = h$. $g^{-1} = v \mod h$. 

\section{SubBytes-Transformation}

Eingabe: Zustand $S_i = \begin{pmatrix}b_{00} & ... & b_{03} \\ . & & . \\ . & & . \\ b_{30} & ... & b_{33} \end{pmatrix}$, $b_{ij}$ Bytes

Jedes Byte aus $S_i$ wird einzeln verändert. Sei $g = (b_7,b_6,...,b_0)$ ein Byte.

\par \medskip

\underline{1.Schritt:} $g \neq 0: g \mapsto g^{-1}, 0 \mapsto 0$ (fasse g als Element in $\mathbb{F}_{2^8}$)

\underline{2. Schritt:} Ergebnis ($c_7,c_6,...,c_0$) nach 1. Schritt wird affin-linearer Transformation unterworden:

\begin{equation}
  (c_0, ..., c_7) \cdot A + b = (d_0, ..., d_7)
\end{equation}

$A = \begin{pmatrix}1 & 1 & 1 & 1 & 1 & 0 & 0 & 0 \\ . & . & . &. &. &. &.&. \\ . & . & . &. &. &. &.&. \\. & . & . &. &. &. &.&. \\ . & . & . &. &. &. &.&. \\ . & . & . &. &. &. &.&. \end{pmatrix}$ 

Übrige Zeilen durch zyklischen Shift der 1. Zeile um 1,2,...,7 Position nach rechts.

\par \medskip

$b = (11000110)$

Ergebnis: $(d_7, ..., d_0)$

SubByes wird in AES durch Table Lookup in einer $16 \times 16$ - Matrix beschrieben:

Einträge sind 0,...,255 (in gew. Reihenfolge).

Byte b = ($b_7,b_6,...,b_0$) bestimmt durch $b_7b_6b_5b_4$ die Zeile und $b_3b_2b_1b_0$ Spalte (Zeichen und Spalten sind mit 0,...,15 nummeriert). Eintrag an der entsprechenden Stelle ist Ergebnis von SubBytes angewendet auf b.

\section{Shift Rows und MixColumns-Transformation}
\begin{itemize}
  \item Shift Rows: Jede der 4 Zeilen nach der $4 \times 4$ - Matrix (nach SubBytes) wird zyklisch nach links verschoben: i-te Zeile um i-1 Stellen (i=1,2,3,4).
  \item MixColumns: Elemente der Eingangsmatrix werden als Elemente in $\mathbb{F}_{2^8}$ betrachtet. Diese Matrix wird von links mit M = $\begin{pmatrix}x & x+1 & 1  & 1\\ 1 & x & x+1 & 1 \\ 1 & 1 & x & x+1 \\ x+1 & 1 & 1 & x \end{pmatrix}$ (über $\mathbb{F}_{2^8}$) über $\mathbb{F}_{2^8}$ multipliziert.

  x entspricht (0, ...,0,1,0)

  1 entspricht (0, .....,0,1)

  M $\begin{pmatrix}a_{00} & . & . \\ a_{10} & . & . \\ a_{20} &. &.  \\ a_{30} &. &. \end{pmatrix} =  \begin{pmatrix} x \cdot a_{00} + x \cdot a_{10} + a_{10} + a_{20} + a_{30} & ... \\  a_{00} + x a_{10} + x a_{20} + a_{20} + a_{30} & ... \\ ... & ... \end{pmatrix}$
\end{itemize}

\section{Rundenschlüsselerzeugung}

Ausgangsschlüssel hat 128 Bit. Wird geschrieben als $4 \times 4$ - Matrix von Bytes, spaltenweise zu lesen.

Spalten $w(0), w(1), w(2), w(3)$.

Definiere 40 weitere Spalten (a 4 Bytes).

w(0),...,w(i-1) seine schon definiert. ($i \ge 4$)

$i \not \equiv 0$ (mod 4): w(i) := w(i-4) $\oplus$ w(i-1) (Komponentenweise XOR-Verkn.)

$i \equiv 0$ (mod 4). w(i) := w(i-4) $\oplus$ T(w(i-1)), wobei T folgende Transformation ist:

\begin{equation}
  w(i-1) = \begin{pmatrix}a & b & c & d \end{pmatrix}, \textnormal{ a,b,c,d Bytes}
\end{equation}

Wende auf b,c,d,a SubBytes an. Liefert e,f,g,h 

\par \medskip

Berechne:

\begin{equation}
  r(i) = (00000010)^{\frac{i-4}{4}}
\end{equation}

Potenz in $\mathbb{F}_{2^8}$.

$i=4, \ x^0 = 1 \leftrightarrow (00000001)$ 

$i=8, \ x^1 = x \leftrightarrow (00000010)$

$i=12, \ x² \leftrightarrow (00000100)$

...

$i=36 \  x^8 \textnormal{ in } F_{2^8}$ reduziert mod $x^8 + x⁴ + x³ + x + 1 = x^4 + x³ + x + 1 \leftrightarrow (00011011)$

\par \medskip

T(w(i-1)) = $\begin{pmatrix}e \oplus r(i) \\ f \\ g \\ h \end{pmatrix}$

\par \medskip

Rundenschlüssel $K_i$ besteht aus Spalten $w(4i),...,w(4i+3)$

\section{Entschlüsselung}

Alle Transformationen in AES sind invertierbar. mit der umgekehrten Reihenfolge der Rundenschlüssel Transformationen invertieren.

\section{Schnelligkeit und Sicherheit}

\begin{itemize}
  \item Schnelligkeit: 

  Software: 200 MBit/sec - 2 GBit/sec

  Hardware: 2GBit/sec - 70 GBit/sec

  \item Sicherheit:

  Nach 2 Runden vollständige Diffusion. 

  Rundenschlüsselerzeugung mit SubBytes verhindert, dass Regelmäßigkeiten im Ausgangsschlüssel fortpflanzen. Verschiednee Ausgangsschlüssel haben nur sehr selten einen Rundenschlüssel gemeinsam.

  AES ist resistent gegen klassische kryptoanalytische Angriffe (differentielle Kryptoanalyse, lineare Kryptoanalyse).

\end{itemize}