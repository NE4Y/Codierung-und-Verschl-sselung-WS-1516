\chapter{Symmetrische Blockchiffren}

\section{Blockchiffre}

Klartext wird in Blöcke von fester Länge n zerlegt. Jeder Block wird einzeln verschlüsselt. Bei festem Schlüssel wird ein und derselbe Block immer in der gleichen Weise verschlüsselt (einfachstes Szenario). 

\textbf{Häufig:} Klartextblocklänge = Chiffretextblocklänge

\par \medskip

R = S = $\{1, 1\}$. Es gibt $2^n$ Blöcke der Länge n.

Jede Blockchiffre ist Permutation dieser $2^n$ Blöcke. Insgesamt $(2^n)!$ viele Blockchiffren der Länge n.


\textbf{Schlüssel:} Permutation

\textbf{Speicherung:} $2^n$ Bider von (0,...,0),...,(1,...,1). Insgesamt $n \cdot 2^n$ Bit.

\subsubsection{Beispiel}

n = 64. Ein Schlüssel erfordert $2^{70}$ Bit $\approx$ 13.5 Millionen Festplatten a 10 TB.

\subsubsection{Lösung}

Es wird nur eine ausgewählte Menge von Permutationen der Blöcke als Schlüsselmenge verwendet.

\section{Lineare Chiffren}

Klartextalphabet = Chiffretextalphabet = $\mathbb{Z}_k$ = $\{0, ..., k-1\}$

\textbf{Klartextblöcke}: Elemente in $\mathbb{Z}_{k}^n$

Lineare Chiffre über $\mathbb{Z}_k$ (Blocklänge n):

m = ($V_1,...,V_n$), $V_i \in \mathbb{Z}_k$, Klartextblock.

\subsubsection{Verschlüsseln}

m $\underbrace{\rightarrow}_{Verschl.} m \cdot A \in \mathbb{Z}_{k}^{n}$, A $n \times n$ - Matrix über $\mathbb{Z}_k$.

($V_1, ..., V_n$) $\cdot$ $\begin{pmatrix}a_{11} & . & . & . & a_{1n} \\ . &  &  &  & . \\ . & & & & . \\ . & & & & . \\ a_{n1} & . & . & . & a_{nn} \end{pmatrix}$ = ($a_{11}V_1 + ...+a_{n1}V_n, ..., a_{1n}V_1 + ... + a_{nn}V_n$) 

Abb $m \rightarrow m \cdot A$ ist invertierbar (d.h. Permutation), wenn A invertierbar ist, das heißt wenn Matrix $A^{-1}$ existiert mit $A \cdot A^{-1}$ = $A^{-1} \cdot A = E_n$.

$c = m \cdot A$.

\newpage

\textbf{Entschlüsselung:}

$c \cdot A^{-1}$ = $m \cdot (A \cdot A^{-1}) = m \cdot E_n = m$.

\par \medskip

Wann ist A invertierbar und wie berechnet man $A^{-1}$?

$\rightarrow$ Detrminante von A $det(A) \in \mathbb{Z}_k$.

$det(a) = a$

$det\begin{pmatrix}a_{11} & a_{12} \\ a_{21} & a_{22}\end{pmatrix} = a_{11} \cdot a_{22} - a_{12} \cdot a_{21}$.

\par \medskip

Inverse zu A (wenn sie existiert):

\begin{equation}
  A^{-1} = \underbrace{(det(A))^{-1}}_{Inveres \ in \ \mathbb{Z}_k} \cdot B, B = (b_{ij}), b_{ij} = (-1)^{i+j} det(A_{ji})
\end{equation}

$A_{ji}$ $(n-1) \times (n-1)$ - Matrix, die aus A durch streichen j-ten Zeile und i-ten Spalte entsteht.

det(A) ist invertierbar in $\mathbb{Z}_k$ $\Leftrightarrow$ ggT(det(A),k) = 1

\par \medskip

m $\rightarrow$ m $\cdot$ A ist invertierbar $\Leftrightarrow$ ggt(det(A),k) = 1.

\subsubsection{Spezialfall n = 2}

ggT($a_{11} \cdot a_{22} - a_{12} \cdot a{21}, k$) = 1

$A^{-1} = (a_{11} \cdot a_{22} - a_{12} \cdot a_{21})^{-1} \cdot \begin{pmatrix}a_{22} & -a_{12} \\ -a_{21} & a_{11}\end{pmatrix}$

\subsection{Beispiel}

$\mathbb{Z}_6$, k = 6, n = 2.

\par \medskip

A = $\begin{pmatrix}0 & 1 \\ 1 & 3 \end{pmatrix}$  det(A) = -1 mod 6 = 5

\par \medskip

ggT(5,6) = 1

\par \medskip

$A^{-1} = 5^{-1} \begin{pmatrix}3 & 5 \\ 5 & 0 \end{pmatrix}$, $5^{-1} = 5$ in $\mathbb{Z}_6$.

\par \medskip

= $5 \cdot \begin{pmatrix}3 & 5 \\ 5 & 0 \end{pmatrix}$.

\par \medskip

\subsubsection{Verschlüsseln von (1,3) mit A}

(1,3) $\cdot \begin{pmatrix}0 & 1 \\ 1 & 3 \end{pmatrix} \underbrace{=}_{mod \ 6} (3,4)$.

\par \medskip

Nachricht (1 3 2 5)

\par \medskip
(2,5) $\begin{pmatrix}0 & 1 \\ 1 & 3 \end{pmatrix} = (5,5)$.

\par \medskip
Chffretext: (3 4 5 5)

\subsubsection{Entschlüsselung:}

(3,4) $\cdot A^{-1} = (3,4) \cdot \begin{pmatrix}3 & 1 \\ 1 & 0 \end{pmatrix} = $(1 3)

\par \medskip

(5.5) $\cdot A^{-1}$ = (2,5).

\par \medskip
\par \medskip
\par \medskip

Anzahl der Schlüssel = Anzahl der invertierbaren $n \times n$ - Matrizen über $\mathbb{Z}_k$.

Für k = 2: ($2^n -1) (2^n-2) (2^n - 2^2) ... (2^n - 2^{n-1})$

n = 64: $\approx 0,29 \cdot 2^{4096}$

\section{Known-Plaintext-Angriff auf lineare Chiffren}

$m \in \mathbb{Z}_{k}^n$, $m \rightarrow m \cdot A$, A invertierbare $n \times n$ - Matrix über $\mathbb{Z}_k$.

Angreifer kennt Klartextblöcke $m_i$ und zugehörige Chiffretextblöccke $c_i$. Er will A ermitteln.

Er benötigt \underline{n} Klartextblöcke $m_1, ...,m_n$  mit zugehörigen Chiffretextblöcken $c_1, ..., c_n$.

Daraus wird er häufig A ermitteln können:

$m_i \cdot A = c_i$, $i=1,...,n$.

\par \medskip

M = $\begin{pmatrix}m_1 \\ m_2 \\ . \\ . \\ m_n \end{pmatrix}$, C = $\begin{pmatrix}c_1 \\ c_2 \\ . \\ . \\ c_n \end{pmatrix}$ $n \times n$ - Matrizen über $\mathbb{Z}_k$.

\par \medskip

M $\cdot$ A = C

Falls M invertierbar ist, so berechne $M^{-1}$.
\tabularnewline
Dann: $M^{-1} \cdot C$ = $M^{-1} (MA) = (M^{-1} M) A = E_n A = A$.

\subsubsection{Beispiel}

n = 2, k = 26

Angneommen wir wissen: KRYPTO $\rightarrow$ QLIPRL mit linearer Chiffre.

$\underbrace{K}_{10} \underbrace{R}_{17} \underbrace{Y}_{24} \underbrace{P}_{15} \underbrace{T}_{19} \underbrace{O}_{14} \rightarrow \underbrace{Q}_{16} \underbrace{L}_{11} \underbrace{I}_{8} \underbrace{P}_{15} \underbrace{R}_{17} \underbrace{L}_{11}$

(n = 2)

\par \medskip

M = $\begin{pmatrix}10 & 17 \\ 24 & 15\end{pmatrix}$, ggT(det(M), 26) = 1 ? Nein, da det(M) gerade.

\par \medskip

Stattdessen:

\par \medskip

M = $\begin{pmatrix}10 & 17 \\ 19 & 14 \end{pmatrix}$, det(M) = ($10 \cdot 14 - 17 \cdot 19$) mod 26 = 25.

\par \medskip

$(det(M))^{-1}$ = -1 mod 26 = 25

$M^{-1} = (-1) \begin{pmatrix}14 & - 17 \\ -19 & 10 \end{pmatrix}$ mod 26

\par \medskip

=  $\begin{pmatrix}12 & 17 \\ 19 & 16 \end{pmatrix}$.

\par \medskip

Zugehöriges C: C = $\begin{pmatrix}16 & 11 \\ 17 & 11 \end{pmatrix}$.

\par \medskip

A = $M^{-1} \cdot C = \begin{pmatrix}12 & 17 \\ 19 & 16 \end{pmatrix} \cdot \begin{pmatrix}16 & 11 \\ 17 & 11 \end{pmatrix} \underbrace{=}_{mod \ 26} \begin{pmatrix}13 & 7 \\ 4 & 21 \end{pmatrix}$

\par \medskip

(10 17) A = (16 11)

(24 15) A = (8 15)

(19 14) A = (17 11)

$\rightarrow$ Test

\section{Bemerkung}

Häufig kann Sicherheit von Blockchiffren erhöht werden durch Hintereinanderausführung mehrerer Blockchiffren. Sinnlos bei linearen Blockchiffren:

2 x lineare Blockchiffre = 1x lineare Blockchiffre

\section{Diffusion und Konfusion}

Shannon, 1949 (Theory of Secrecy Systems)

Kriterien für gute Chiffrierverfahren.

\subsection{Diffusion}

Statistische Aufälligkeiten im Klartext sollen im Chiffretex nicht mehr erkennbar sein.

\textbf{Insbesondere:} Jedes Chiffretextzeichen muss von mehreren Klartextzeichen abhängen.

\subsection{Konfusion}

Aus statistischen Eigenschaften des Chiffretextes soll nicht auf den Schlüssel zurückgeschlossen werden.

\textbf{Insbesondere:} Jedes Chiffretextzeichen hängt von mehreren Schlüsselzeichen ab.

\par \medskip


Lineare Chiffren: gute Diffusion, halbwegs gute Konfusion.

\section{Feistel-Chiffren}

Horst Feistel, IBM (1915-1990)

1971: Konstruktionsprinzip für symmetrische Blockchiffren. Wichtigste Realisierung: DES (Data Encryption Standard)

\par \medskip

\textbf{Prinzip:}

\begin{enumerate}
  \item Folge von Blocksubstitutionen und Blocktranspositionen (= Permutation der Einträge in einem Block). (Hohe Diffusion und Konfusion).
  \item Erzeugung von Rundenschlüsseln aus Ausgangsschlüssel.
\end{enumerate}

$m \in \mathbb{Z}_{m}^{n}$, n gerade

\Tree [.\text{Klartextblock m} [.$L_0$ ] $R_0$ ]

Bild kommt von Maxi

m = ($L_0, R_0$)

$L_i = R_{i+1}$

$R_i = L_{i-1} \oplus f_{k_i}(R_{i-1})$

$i=1,...,r-1$

$L_r = L_{r-1} \oplus f_{k_r}(R_{r-1})$

$R_r = R_{r-1}$

Konkrete Realisierung hängt von f ab + Art der Rundenschlüsselerzeugung.

\par \medskip

Die Abbildungen $R_{i-1} \rightarrow f_{k_i}(R_{i-1})$ dürfen nicht affin-linear sein. (affin-linear: x $\mapsto xA +b$, Hintereinanderausführung affin-linearer Chiffren ist wieder affin-linear)

Denn sonst ist die gesamte Feistel-Chiffre affin-linear und angreifbar ähnlich wie lineare Chiffren.

\par \medskip

\textbf{Entschlüsselng} einer Feistel-Chiffre = \textbf{Verschlüsselung} mit der umgekehrten Reihenfolge der Rundenschlüssel.