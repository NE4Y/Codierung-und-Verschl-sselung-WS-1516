\chapter{Signaturen, Hashfunktionen, Authentifizerung}

  \section{Digitale Signaturen}

  Anforderung: Niemand außer A kann Dokument mit der Signatur von A versehen, selbst wenn er Signatur von A von anderen Dokumenten kennt.

  \par \medskip

  Also auch: A kann nicht abstreiten, Dokument signiert zu haben.

  \par \medskip

  \par \medskip

  Signatur gewährleistet:

  \begin{itemize}
    \item \underline{Identitätseigenschaft} des Unterzeichners des Dokuments

    \item \underline{Echtheitseigenschaft} des Dokuments
  \end{itemize}

  Außerdem: \underline{Verifikationseigenschaft}

  \par \medskip

  Jeder Empfänger einer von A signierten Nachricht muss Signatur verifizieren können.

  \par \medskip

  \par \medskip

  Realisierung über RSA-Signatur.

  \section{RSA-Signatur}

  A will Dokument signieren. Sie besitzt öffentlichen RSA-Schlüssel (n,e), geheimen RSA-Schlüssel d.

  \par \medskip

  Signatur von Dokument m ($m < n$):

  \begin{center}
    s(m) := $m^d \mod n$
  \end{center}

  A sendet (m, $s_A(m)$) an B.

  B verfiziert: $s_A(m)^e \mod n = m^{de} \mod n = m$, Signatur wird akzeptiert.

  \par \medskip

  Falls $s_A(m)^e \mod n \neq m$, Signatur wird nicht akzeptiert.

  \par \medskip

  \par \medskip

  Funktioniert so gut, da bei RSA Verschlüsselung / Entschlüsselung vertauschbar sind.

  \par \medskip

  Signaturen sind auch mit ElGamal konstruierbar, komplizierter wegen Zufallswahl bei der Verschlüsselung.

  \par \medskip

  Problem: Signatur ist so lang wie Nachricht.

  \par \medskip

  Ausweg: Kryptografische Hashfunktionen.

  \section{Definition Hashfunktion}

  R endliches Alphabet. \underline{Hashfunktion} ist Abbildung $R^* \rightarrow R^k$ (k fest)

  \section{RSA-Signatur mit Hashfunktion}

  A will m signieren (jetzt nicht notwendig $m < n$). Hashfunktion H der Länge k (Hashwerte $< 2^k \le n$).

  H öffentlich bekannt.

  Sie sendet (m, $\underbrace{H(m)^d \mod n}_{s_A(m)}$)

  \par \medskip

  Verifikation: B bildet H(m) aus m mit H. $s_A(m)^e \mod n = H(m)$? 

  $\rightarrow$ Ja $\checkmark$, Nein: nicht akzeptiert

  \section{Anforderung an H}

  \begin{enumerate}[label=(\alph*)]
    \item Angreifer kennt $(m, H(m)^d \mod n)$.

    Er kann H(m) bestimmen. Gelingt es ihm ein $m' \neq m$ zu finden mit $H(m') = H(m)$, so ist $(m', H(m)^d \mod n)$ gültige Signatur von $m'$ durch A.

    \item Angreifer wählt zufällig $y$ und berechnet $y^e \mod n = z$. Gelingt es ihm ein $m$ zu finden mit $H(m) = z$, so ist $(m,y)$ eine gültige Signatur von m durch Alice.

    Verifikation:

    \begin{itemize}
      \item $y^e \mod n = z$ $\checkmark$
      \item $H(m) = z$ $\checkmark$
    \end{itemize}
  \end{enumerate}

  \section{Definition: Kryptographische Hashfunktion}

  Eine \underline{kryptographische Hashfunktion} ist Hashfunktion H, die folgende Bedingungen erfüllen muss:

  \begin{enumerate}
    \item[(1)] H ist \underline{Einwegfunktion} (um Angriffe vom Typ 6.5.b) zu unterbinden)
    \item[(2)] H ist \underline{schwach kollisionsressistent}, d.h. zu gegebenem n ist es \underline{nicht effizient} möglich ein $m' \neq m$ zu finden mit $H(m') = H(m)$ (um Angriffe vom Typ 6.5.a) zu unterbinden) (second pre-image resistant)     
  \end{enumerate}

  Verschärftere Anforderung an H als (2):

  \begin{enumerate}
    \item[(2')] H ist \underline{stark kollisionsressistent}, d.h. es ist nicht effizient möglich zwei $m_1 \neq m_2$ zu finden mit $H(m_1) = H(m_2)$
  \end{enumerate}

  (Solche Kollisionen gibt es, sogar unendlich viele)

  \par \medskip

  [Kryptographische Hashfunktion = message digest = digital fingerprint]

  \par \medskip

  \par \medskip

  \subsubsection*{Wie lang sollte H(m) sein?}

  \begin{itemize}
    \item Nicht zu lang (Sinn von Hashfunktion)
    \item Nicht zu kurz (``Geburtstagsparadox'') $\rightarrow$ Geburtstagsattacke
  \end{itemize}

  (H(m) hat 128 Bit Länge. Angriff auf starke Kollisionsressistenz:

  Wenn er $2^{128} +1$ H(m)'s erzeugt hat, hat er sicher Kollision)

  \section{Satz (Geburtstagsparadox)}

  Ein Merkmal komme in m verschiedenen Ausprägungen vor. Jedes Objekt (einer Grundgesamtheit) besitze genau eine dieser Merkmalesausprägungen (mit gleicher Wahrscheinlichkeit).

  \par \medskip

  Ist dann $l \ge \frac{1+ \sqrt{1+8 \cdot m \cdot ln(2)}}{2} (\approx 1.18 \sqrt{m})$, so ist die Wahrscheinlichkeit, dass unter l Objekten zwei die gleiche Merkmalsausprägung haben, $\ge \frac{1}{2}$.

  (Geburtstagsparadoxon: m = 366, $l \ge 23 \rightarrow$ Wahrscheinlichkeit $\ge \frac{1}{2}$, dass 2 Personen am gleichen Tag Geburtstag haben)

  \subsection{Beweis}

  Gegeben l Objekte.

  Alle Ereignisse: ($g_1, ..., g_l) \in \{1,2,...,m\}^l$)

  Anzahl: $m^l$.

  Alle $g_i$ paarweise verschieden : $\prod_{i=0}^{l-1} (m-i)$

  \par \medskip

  Wahrscheinlichkeit, dass keine 2 Objekte gleiche Merkmalsaussprägung haben:

  \begin{center}
    $q := \frac{\prod_{i=0}^{l-1} (m-i)}{m^l} = \prod_{i=0}^{l-1} (1- \frac{i}{m})$
  \end{center}

  Benutze: $e^x \ge 1+x$.

  \begin{equation*}
    q \le \prod_{i=0}^l e^{- \frac{i}{m}} = e^{\sum_{i=0}^{l-1} (- \frac{i}{m})} = e^{- \frac{1}{m} \cdot \sum_{i=0}^{l-1} i} = e^{- \frac{l(l-1)}{2m}}
  \end{equation*}

  Wann ist $q \le \frac{1}{2}$?. Sicher, wenn

  $e^{- \frac{l(l-1)}{2m}} \le \frac{1}{2} \Leftrightarrow \frac{l(l-1)}{2m} \ge ln(2) \leftrightarrow l^2 - l - 2m \cdot ln(2) \ge 0 \Leftrightarrow l \ge \frac{1 + \sqrt{1+8m \cdot ln(2)}}{2}$

  \section{Geburtstagsattacke (gegen starke Koll.res.)}

  Gegeben: H: $\{0,1\}^* \rightarrow \{0,1\}^k$

  Angreifer erzeugt möglichst viele Hashwerte., diese werden auf Kollisionen untersucht. 

  Wende 6.7 an: 

  \begin{itemize}
    \item Menge der Objekte = $\{0,1\}^*$
    \item Merkmalausprägungen = Hashwerte
    \item Anzahl aller möglichen Hashwerte m = $2^k$
  \end{itemize}

  Wahrscheinlichkeit $~$ $\frac{1}{2}$, dass Kollision vorliegt, bei $~ \sqrt{2^k} = 2^{\frac{k}{2}}$ vielen erzeugten Hashwerten.

  \par \medskip

  \par \medskip

  k = 64 keine ausreichende Sicherheit. Wahrscheinlichkeit $\frac{1}{2}$ Kollision bei $2^{32}$ $\approx$ $4 \cdot 10^9$ erzeugten Hashwerten

  \par \medskip

  Forderung: $k \ge 128$, besser $k \ge 160$

  \section{Bemerkung}

  Weit verbreitete Hashfunktionen:

  \begin{itemize}
    \item MD5 (R. Rivest 1992) n = 128
    \item SHA-1 (NSA, NIST 1992/93/95) n = 160
    \item SHA-2-Familie (SHA-224, 256, 384, 512 NIST 2002/04)
  \end{itemize} 

  Kollision bei SHA-1 mit ca. $2^{51} - 2^{57}$ Hashoperationen (vgl. mit $2^{80}$)

  \par \medskip

  2007: NIST - Ausschreibung für neuen Stadard bei Hashfunktionen

  2012: Gewinner KECCAK (Bertoni, Daemen, Peeters, van Assche). Länge des Hashwertes: 224 - 512 Bit 

  Konstruktionsprinzipien: Paar, Pelzl, Kap. 11

  \section{Authentifizierung}

  Nachweis / Überprüfung der Identität.

  Forderung: Niemand anderes als A darf sich als A ausgeben können (auch nicht der Verifizierer).

  \par \medskip

  Authentifizierung durch:

  \begin{itemize}
    \item Wissen
    \item Besitz
    \item biometrische Merkmale
  \end{itemize}

  \section{Passwörter}

  A wählt Passwort w. Bei B, gegenüber dem sich A authentifizieren will, ist f(w) gespeichert, f Einwegfunktion (z.B. Hashfunktion).

  \par \medskip

  $\rightarrow$ Unsicher !

  \par \medskip

  \par \medskip

  Besser: Einmal-Passwörter

  \par \medskip

  \par \medskip

  Lamport, 1981:

  f Einwegfunktion (öffentlich bekannt)

  A: $w_A$, $f(w_A), f^2(w_A) = f(f(w_A)), ..., f^n(w_A)$

  Am Anfang sendet A an B: $w_0 = f^n(w_A)$ (auf sicherem Weg)

  \newpage

  Authentifizierung: 

  \begin{itemize}
    \item A sendet $w_1 = f^{n-1}(w_A)$ an B

    \item B testet, ob $f(w_1) = w_0$.

  \end{itemize}

  Nach der ersten Authentifizierung ersetz B $w_0$ durch $w_1$. Nächste Authentifizierung $A \rightarrow B$ $f^{n-2}(w_A) = w_2$, B: $f(w_2) = w_1$. Etc.

  \par \medskip

  \par \medskip

  Vorteil: Aus $f^i(w_A)$ lässt sich $f^{i-1}(w_A)$ nicht rekonstruieren.

  \section{Challenge-Response-Authentifizierungen}

  Signatur von Zufallsstring, z.B. mit RSA-Verfahren. 

  $(n,e)$ öffentlicher Schlüssel von A, $d$ geheimer Schlüssel von A)

  $A \overset{auth.}{\rightarrow} B$

  B wählt Zufallszahl $r < n$. A berechnet $r^d \mod n = c \rightarrow B$.

  B verfiziert, ob $c^e \mod n = r$.

  \par \medskip

  \par \medskip

  \textbf{Wichtig:} Authentifizierungsschlüssel und Verschlüsselungsschlüssel müssen verschieden sein !