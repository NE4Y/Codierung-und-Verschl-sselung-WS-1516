\section{Grundbegriffe und einfache Verfahren}

\textbf{Klartext}: Unverschlüsselter Text (plain text)

\textbf{Chiffre}: Verschlüsselter Text (cipher text)

\textbf{Sender Alice}: Schlüssel $\rightarrow$ Verschlüsselung

\textbf{Empfänger Bob}: Schlüssel $\rightarrow$ Entschlüsselung

\par \medskip

Encryption über Alphabet R, decryption über Alphabet S.

Verschlüsselung erfordert:

\begin{itemize}
  \item Verschlüsselungsverfahren (Funktion E)
  \item Schlüssel $k_e$ (encryption key) aus Menge $K$ von möglichen Schlüsseln
\end{itemize}

E(m, $k_e$) = c, mit m Klartext, $k_e$ Schlüssel und c Chiffre-text.

Entschlüsselung erfordert

\begin{itemize}
  \item Entschlüsselungsfunktion (Funktion D)
  \item Schlüssel $k_d$ (decryption key, hängt ab von $k_e$)
\end{itemize}

D(c, $k_d$) = m

Für festes $k_e$ soll die Funktion E(., $k_e$) injektiv sein, d.h. 

\begin{equation}
m_1 \neq m_2 \rightarrow E(m_1,k_e) \neq E(m_2, k_e)
\end{equation}

D(., $k_d$) = E(., $k_e$)$^{-1}$

\subsection{Definition}
Ist $k_e = k_d$ (oder falls $k_d$ einfach aus $k_e$ zu berechnen ist), so spricht man von \underline{symmetrischen Verschlüsselungsverfahren}.

\par \medskip

Ist $k_d$ nicht oder nur mit großem Aufwand aus $k_e$ berechenbar, so spricht man von \underline{asymetrischen Verfahren}.

\par\medskip

Im zweiten Fall kann man $k_e$ auch veröffentlichen: Public-Key-Verfahren. Sicherheit eines Verschlüsselungsverfahrens darf nur von der Geheimhaltung von $k_d$ abhängen. Bei symmetrischen Verfahren muss $k_e = k_d$ auf sicherem Wege ausgetauscht werden.

\par\medskip

Asymmetrische Verfahren: 

Bob erzeugt $k_e$, $k_d$.

\subsection{Anzahl Schlüssel symetrisch vs. asymetrisch}
Für symmetrische Verfahren gilt:

\begin{equation}
  {n \choose 2 } = \frac{n \cdot (n-2)}{2} = O(n²) 
\end{equation}

Für asymmetrische Verfahren gilt:

\begin{equation}
  2n = O(n)
\end{equation}


\section{Beispiel}

\begin{enumerate}[label=(\alph*)]
  \item R = S = $\{0,1,...,25\}$

  Verfahren: Verschiebechiffre

  Menge der Schlüssel: K = $\{0,1,...,25\}$

  Elemente aus R werden einzeln verschlüsselt:

  Wähle Schlüssel $i \in K$

  \textbf{Verschlüsseln:}  $x \in R \mapsto x+i$ mod 26

  \textbf{Entschlüsseln:} $y \mapsto y-i$ mod 26

  $m = x_1 ... x_r$, $x_j \in R$

  $E(m,i)$ = (($x_1 +i$) mod 26) (($x_2 +i$) mod 26) ... (($x_r + i$) mod 26) = c

  $D(c,i)$ = ($y_1-i$) mod 26) ... (($y_2 -i$) mod 26) = m 

  Verfahren unsicher, da K klein.

  \item Verallgemeinerung: zeichenweise Substitutionschiffre

  R = S = $\{0, ..., 25\}$

  K = Menge aller Permutationen von R

  Wähle Schlüssel $\pi \in K$, m = $x_1 ... x_r$, $x_j \in R$

  \textbf{Verschlüsseln:} 

  $E(m, \pi) = \pi(x_1) ... \pi(x_r) = c$

  \textbf{Entschlüsseln:}

  $D(c, \pi) = \pi^{-1}(y_1) ... \pi^{-1}(y_r) = m$

  $|K| = 26! \approx 4 \cdot 10^{26}$

  Angenommen $10^{12}$ Schlüssel pro Sekunde testbar.

  Angenommen 50$\%$ Schlüssel werden getestet. 

  Man benötigt dazu: $2 \cdot 10^{14}$ Sekunden $\approx$ 6.000.000 Jahre.
\end{enumerate}

\section{Übersprungen}
\section{Übersprungen}

\section{Prinzip von Kerkhoffs (1883)}

Sicherheit eines Verschlüsselungsverfahrens darf nur von der Geheimhaltung des Schlüssels $k_d$ abhängen, nicht von der Geheimhaltung des Verschlüsselungsalgorithmus ($\Rightarrow$ sollte offen gelegt werden: Praxistest).

\section{Kryptoanalyse}

Qualitative Unterschiede: 

\begin{itemize}
  \item Schlüssel $k_d$ lässt sich ermitteln
  \item Ermittlung einer zu D(., $k_d$) äquivalenten Funktion (evtl. nur für gewisse $k_e$)
  \item Finden des Klartextes für einen speziellen Chiffretext
\end{itemize}

\subsection{Arten von Angriffen}

\subsubsection{Ciphertext-only-Angriff}

Lediglich der Chiffretext ist dem Angreifer bekannt.

\subsubsection{Known-Plaintext-Angriff}

Der Angreifer kennt bereits Plaintext - Chiffretext Kombinationen.

\subsubsection{Chozen-Plaintext-Angriff}

Der Angreifer kann sich mit einem selbstgewählten Plaintext Chiffretexte erzeugen.

\subsubsection{Chozen-Ciphertext-Angriff}

Der Angreifer kann sich gewünschte Chiffretexte entschlüsseln lassen.


\chapter{One-Time-Pad und perfekte Sicherheit}

\section{Lauftextverschlüsselung}

Früher im militärischen Bereich \underline{Lauftextverschlüsselungen}.

\textbf{Klartext}: Endliche Folge z.B. über $\{0, ..., 25\}$.

Addiere (mod 26) anderen Text (z.B. ab gewisser Stelle im Buch).

\par \medskip

\textbf{\underline{Problem}}: Häufige Buchstaben treffen bei Addition häufig auf Buchstaben $\Rightarrow$ kryptoanalytische Möglichkeiten.

\section{One-Time-Pad}

Klartextalphabet R = $\mathbb{Z}_2 =  \{0, 1\}$. 

\par \medskip

\textbf{Verschlüsselung:} Addiere zu Klartext der Länge n eine binäre Zufallsfolge der Länge n (Addition mod 2 = XOR).

\par \medskip

\textbf{Ergenibs:} Chiffretext
\par \medskip


$\overbrace{m}^{Klartext}$ $\oplus$ $\overbrace{k}^{Zufallsfolge}$ = c

\par \medskip

\textbf{Entschlüsseln:} $c \oplus k = m \oplus k \oplus k = m$

\subsubsection{One-Time-Pad}
k darf nur einmal verwendet werden. 

$m_1  \oplus k = c_1$, $m_2 \oplus k = c_2$

$c_1 \oplus c_2 = m_1 \oplus k \oplus m_2 \oplus k = m_1 \oplus m_2$

\section{Was ist Zufallsfolge der Länge n?}

Frage ist sinnlos! Es kommt auf die Erzeugung an. Folge von Nullen und Einsen, die so erzeugt werden, dass jeses Bit unabhängig von den vorhergehenden mit Warscheinlichkeit $\frac{1}{2}$ erzeugt wird. 

Jede Folge der Länge n hat die Wahrscheinlichkeit $\frac{1}{2^n}$. 

\section{One-Time-Pad ist perfekt sicher}

\textbf{Gegeben:} Chiffretext c.

Dann gilt: $pr(m|c) = \overbrace{pr(m)}^{A-priori\ W-keit}$, für alle Klartexte m.

\par \medskip

$pr(m|c)$ = Wahrscheinlichkeit, dass m Klartext war, wenn ich c kenne.

\par \medskip

Substitutionschiffre ist nicht perfekt sicher:

\textbf{Chiffretext:} OCKTT

$\Rightarrow$ m = HAUCK kann nicht Klartext gewesen sein. $pr(m|c) = 0$